\documentclass[a4paper, magyar]{article}

\usepackage[utf8]{inputenc}
\usepackage[magyar, english]{babel}
\usepackage[T1]{fontenc}
\usepackage{testhyphens}

%opening
\title{Deep learning alapú szótagolás}
\author{N{\'e}meth Gergely D{\'a}niel}
\date{}

\begin{document}
\fontfamily{ptm}\selectfont
\maketitle



\selectlanguage{english}
\begin{abstract}

\end{abstract}
\selectlanguage{magyar}
\begin{abstract}
	
\end{abstract}

\section{Bevezetés}
\section{Szótagoló programok}
Az elterjedt szótagoló algoritmusok kétféle csoportba bonthatóak: szabály- vagy szótáralapúak. A szabad szoftverek világában \textit{de facto} a \TeX szótagolási algoritmusát használják.

A \TeX\,eredeti szótagoló algoritmusát Prof. Knuth tervezte 1977 nyarán\cite{knuth1979tex}. Ez három fő szótagolási szabályt alkalmazott: $1)$ utótag leválasztás, $2)$ előtag leválasztás és $3)$ magánhangzó - mássalhangzó - mássalhangzó - magánhangzó (vccv) elválasztás, azaz ha ilyen betűnégyes található a szóban, legtöbb esetben a mássalhangzók mentén elválasztható. Ez a három szabály gyakran alkalmazható, azonban már az első algoritmusban kiegészült kisebb szabályokkal (,,break vowel-q" vagy ,,break after ck") és kivételek listájával(300 szó).

A \TeX82 verzióhoz megjelent Liang szótagoló algoritmusa\cite{liang1983word}, aminek legfontosabb újítása a minta(\textit{patterns}) alapú szótagolás bevezetése volt. Ennek lényege, hogy a szótagolási szabályok mintákra definiálódtak és az algoritmus a szótagolás során ezeket a mintákat keresi a szóban.

A \TeX jelenlegi verziójában a Hunspell szótagoló algoritmust alkalmazzák\cite{nemeth2006automatic}. Ez az eljárás Liang algoritmusán alapszik, amit nyelvfüggő speciális szótagolási kiterjesztésekkel (\textit{non-standard hyphenation extension}) egészít ki.
\subsection{Liang algoritmusa}
Liang szótagoló algoritmusának alapját a szótagolási minták alkotják\cite{liang1983word}. Az algoritmust az angol 
{\fontfamily{pcr}\selectfont
	hyphenation%
} elválasztását az alábbi módon végzi:

Az algoritmus először megnézi, hogy a szó benne van-e a kivételek listájában (ez lényegében teljes szavakat tartalmaz mintaként). A \textit{hyphenation} szó nincs a kivételek listájában.

Ezután a szó elejére és végére illeszt egy pont  jelző karaktert. Ennek jelentősége azoknál a mintáknál lesz, amelyek akkor érvényesülnek, ha a szó elején, vagy a végén szerepelnek.
{\fontfamily{pcr}\selectfont
	\begin{center}
	.hyphenation.
	\end{center}
}
Ezt követően a minták között keres illeszkedést. A \textit{hyphenation} szóra ezek az alábbiak: 
{\fontfamily{pcr}\selectfont hy3ph, he2n, hena4, hen5at, 1na, n2at, 1tio,  2io} \cite[37.\ oldal]{liang1983word}
A megfelelő mintákat ráillesztve a szóra, a bennük szereplő számokat a szó karakterei közé szúrva az \ref{liang-hyp}. ábrán szereplőket kapjuk.%
\footnote{Az ábrázolásmód N{\'e}meth cikkéből származik\cite{nemeth2006automatic}.}
\begin{figure}[h]\centering
	{\fontfamily{pcr}\selectfont
		\setlength{\tabcolsep}{0pt}
		\begin{tabular}{rrrrrrrrrrrrr}
			.& h& y& p& h& e& n& a & t& i& o& n& .\\
			 & h& y&3p& h\\
			 &  &  &  & h& e&2n\\
			 &  &  &  & h& e& n& a&4\phantom{t}\\
			 &  &  &  & h& e& n&5a&t\\
			 &  &  &  &  &  &1n& a\\
			 &  &  &  &  &  & n&2a& t\\
			 &  &  &  &  &  &  &  &1t& i& o\\
			 &  &  &  &  &  &  &  &  &2i& o\\
			 \hline
			 .&0h&0y&3p&0h&0e&2n&5a&4t&2i&0o&0n&0.\\
			  & h& y&-p& h& e& n&-a& t& i& o& n
		\end{tabular}
	}
\caption{A \textit{hyphenation} szótagolása}\par\medskip\centering
\label{liang-hyp}
\end{figure}


Az algoritmus következő lépésében minden két szomszédos karakter közé illesztünk egy számot. Alapértelmezetten $0$-t és ha ennél nagyobb számot találtunk a mintaillesztésnél, a legnagyobb kapott értéket adjuk meg.

A szótagolás szabálya innen már csak egy lépés: a páratlan számok mentén elválasztunk, a párosoknál nem. Ezzel megkaptuk a
{\fontfamily{pcr}\selectfont
	hy-phen-ation%
}
elválasztást.
\subsubsection{Minták választása}
A fenti algoritmus hatékonyságát nyilvánvalóan a minták mennyisége és hatékonysága határozza meg. Csak azok a szavak választódnak el, amelyekre illeszkedik minta és csak azok lesznek jó elválasztások, melyekre jó minta illeszkedik.

\subsection{Hunspell}
\subsubsection{Szótagolási hibák a magyar Hunspellben}
\begin{minipage}{0.5\textwidth}
	Hunspell szótagolása:
\begin{checkhyphens}
	autóval
\end{checkhyphens} 		
\end{minipage}
\begin{minipage}{0.5\textwidth}
	Helyesen:
	
	a-u-tó-val
\end{minipage}


\section{Deep learning ismertető}

\bibliographystyle{plain}
\bibliography{references}

\end{document}
